\documentclass[../main.tex]{subfiles}

\begin{document}

\chapter{Discussion}
\label{sec:discussion}
This chapter will be divided, like the results chapter, into two sections. 
The first chapter discusses the pre-training results where the results from the experiment and the pre-processing of the data is discussed.
The second chapter discusses the performance results of the trained model.
This includes a further discussion about why the model might generalize well for some cases but bad for others.

\section{Pre-training results}
\label{sec:discussion_pre-training-results}
The \ac{EMG} pattern from most of the muscles was shown to highly correlate between subjects.
With that stated, it should be emphasized that the subjects were all young, healthy adults.
Many have researched difference in \ac{EMG} signal between various target groups and varying conditions \cite{Rezgui2013,Sacco2010}.
From these researches, it can be concluded that \ac{EMG} signal can vary in amplitude, lag, and/or pattern depending on pathology and conditions.

The \ac{EMG} from gluteus maximus was observed to vary between subjects, much more than the \ac{EMG} from other muscles.
There could be a number of reasons for this discrepancy.
For example, differing location of the sensor.
The researcher found it more difficult to determine the correct sensor location for the gluteus maximus than the other muscles.
It should also be mentioned that the sensor was the only sensor that was located underneath the clothes (shorts) of the user.
The shorts could have got caught in the sensor during walking, possibly pulling on it.
Furthermore, underneath the clothing there could be more perspiration, which could affect the signal.
More specifically, sweat has been shown to dampen the signal \cite{Abdoli-Eramaki2012}.


\section{Post-training results}
\label{sec:discussion_post-training-results}
Add text about how the models compare
When looking at the results from section \ref{sec:post-training-results}, it is clear that the same model does not perform as well for every case.


The design of the model had some iterations. 
One notable design criteria, that changed throughout the process, is the number of timesteps included in the \ac{LSTM} model.
Looking at 20 past samples of \ac{EMG} results in a rather deep network when considering running it on an embedded system.
At first, much fewer timesteps were used.
Those models performed alright in cases where the testing set was a part of the same session's dataset as the training set.
However, they completely failed to generalize between subjects.
It was not until using 20 timesteps that the the error on the test set was observed to decrease during the training of the model.

\section{Real-time compatability}
\label{sec:discussion_real-time}
When used in an exoskeleton, the assistance (or resistance) of the exoskeleton may affect the \ac{EMG} signal.
Since the model is based on the 20 past \ac{EMG} signals, this may create a bad feed-back loop.
For example, if the model uses \ac{EMG} signals from $t-20$ to $t-1$ to make the exoskeleton assist knee-flexion at $t$ then this could affect the \ac{EMG} signal collected at $t$.
This \ac{EMG} signal may be contradicting intended movement, and thus the prediction at $t+1$ might not be correct.
This in turn has more affect on the user's motion and their \ac{EMG} signal, thus resulting in the bad feed-back loop.

\end{document}