\documentclass[../main.tex]{subfiles}

\begin{document}

\chapter{Discussion}
\label{sec:discussion}
This chapter will be divided, like the results chapter, into two sections. 
The first chapter discusses the pre-training results where the results from the experiment and the pre-processing of the data is discussed.
The second chapter discusses the performance results of the trained model.
This includes a further discussion about why the model might generalize well for some cases but bad for others.

\section{Pre-training results}
\label{sec:discussion_pre-training-results}

The normalized \ac{EMG} signal from gluteus maximus on subject 2 is much lower than from the other subjects and no clear pattern is seen from the average.
However, when looking closely, the standard deviation shows that a similar pattern might be found in some of the cycles.
\ac{EMG} signal from gluteus maximus can often be rather low due to high isolation between muscle and skin by adipose tissue.
The relative amplitude of the signal can thus be very sensitive to the placement of the sensor.
As explained in detail in appendix \ref{sec:A-EMG}, the \ac{EMG} signal can vary highly between sessions due to many factors, including placement of the sensor.
Incidentally, data from subject 2 was collected in two separate sessions where one session contained much fewer trials than the other.
The shorter session, for some reason, also had much higher amplitude in \ac{EMG} signal from gluteus maximus.
This explains why the average of the normalized values of the combined dataset from both sessions is near zero.
The pattern in the standard deviation is the result of the signal from the shorter (higher amplitude) session.

\end{document}