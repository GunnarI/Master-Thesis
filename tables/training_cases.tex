\begin{table}[ht!]
    \centering
    \caption{Dataset combination cases for the model training and testing. The cases test various abilities of the model to generalize when trained and tested on different datasets.}
    \begin{tabular}{>{\raggedright}p{0.45\textwidth} | p{0.45\textwidth}}
    \multicolumn{2}{>{\raggedright}p{0.9\textwidth}}{\textbf{Case 1}: Ability to generalize across sessions.} \\ \hline
        \textbf{Training set} & \textbf{Testing set} \\
        Subject06/Session1 & Subject06/Session2 \\ \hline
    \multicolumn{2}{>{\raggedright}p{0.9\textwidth}}{\vspace{-4pt}\textbf{Case 2}: If added data from other subjects improve case 1.} \\ \hline
        \textbf{Training set} & \textbf{Testing set} \\
        Subject01/Both & Subject06/Session2 \\ 
        Subject02/Both & \\
        Subject06/Session1 & \\ \hline  
    \multicolumn{2}{>{\raggedright}p{0.9\textwidth}}{\vspace{-4pt}\textbf{Case 3}: Ability to generalize across subjects.} \\ \hline
        \textbf{Training set} & \textbf{Testing set} \\
        Subject01/Both & Subject02/Both \\ 
        Subject06/Both & \\ \hline  
    \multicolumn{2}{>{\raggedright}p{0.9\textwidth}}{\vspace{-4pt}\textbf{Case 4}: Same as case 3 but switched subjects for train and test} \\ \hline
        \textbf{Training set} & \textbf{Testing set} \\
        Subject02/Both & Subject01/Both \\ 
        Subject06/Both & \\ \hline 
    \multicolumn{2}{>{\raggedright}p{0.9\textwidth}}{\vspace{-4pt}\textbf{Case 5}: Ability to generalize when trained on many subjects} \\ \hline
        \textbf{Training set} & \textbf{Testing set} \\
        Random 80\% of all the usable data & Random 20\% of all the usable data (different data than the training set) \\ \hline 
    \end{tabular}
    \label{tab:training-cases}
\end{table}