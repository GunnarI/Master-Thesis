\documentclass[../main.tex]{subfiles}

\begin{document}

\chapter{Introduction}
The concept of \textit{exoskeletons} as it will be used in this thesis refers to \textit{robotic exoskeletons} or \textit{powered exoskeletons}. 
They can be defined as mobile, computer controlled, electromechanical structure which can be worn by a human to provide power amplification, support and/or assistance in human motion \cite{Anam2012, Gorgey2018}. 
Exoskeletons have been used in rehabilitation for individuals that have suffered a stroke or a \ac{SCI}, or individuals with \ac{CP} \cite{Gorgey2018}.
They are suggested to give the patients assistance in gait training so that the patients can endure increased amount of training.
However, exoskeletons often provide passive movement to the body without any need of muscle contraction, limiting the physical activity of the patient \cite{Gorgey2018}.
For the purpose of 
Referring to assistance in the knee joint, an electric motor can be placed at the joint, connected to the supporting structure.

A typical powered exoskeleton would have a motor or some sort of actuator to produce torque at the joints (e.g. knee joint) to enable or assist the user.

The amount of torque produced for any given motion depends on the control system of the exoskeleton.
Many different methods have been researched and implemented towards the goal of making a good control system that provides exactly the assist that the user wants and/or needs.
The key to that goal is to estimate or predict the joint torque associated with a certain motion performed by the user.

Exoskeletons are .....

% What to talk about:
% Talk about exoskeletons:
% \begin{itemize}
%     \item Briefly how they assist with motion
%     \item Human-robot interaction
%     \item Smoothness of real-time control
%     \item Myoelectric control, parametric (Hill-type) vs. non-parametric (neural networks) \cite{Anam2012}.
% \end{itemize}

% Discuss limitations for using real-time analysis, e.g. data acquisition, complexity of analysis and processing power of microcontrollers on the exoskeleton.

% Discuss ANN together with EMG as an alternative to motion capture and complex analysis such as inverse dynamics and Hill-type muscle models. 
% Also briefly mention what ANN is all about and its limitations, specially in relation to overfitting and underfitting.



%We use the \emph{biblatex} package to handle our references.  We therefore use the
%command \texttt{parencite} to get a reference in parenthesis, like this
%\parencite[12]{Pizzolato2015}\cite{Pizzolato2015}.  It is also possible to include the author
%as part of the sentence using \texttt{textcite}, like talking about
%the work of \textcite{Lloyd2003}.

\section{Research Question}
Is \ac{ANN} feasible to use as a myoelectric control model in an exoskeleton and how does it perform compared to joint moment prediction in relation to inverse dynamics. 
Is it possible to generalize \ac{EMG} to torque prediction across different conditions of the same subject or even across different subjects doing the same task.

\end{document}