\documentclass[../main.tex]{subfiles}

\begin{document}

\chapter{Introduction}
The concept of \textit{exoskeletons} as it will be used in this thesis refers to \textit{robotic exoskeletons} or \textit{powered exoskeletons}. 
They can be defined as mobile, computer controlled, electromechanical structure which can be worn by a human to provide power amplification, support and/or assistance in human motion \cite{Anam2012, Gorgey2018}. 
Lower limp exoskeletons have been used in rehabilitation for motion assistance.
Individuals that suffer from neurological/neuromuscular disorders, such as after a stroke or \ac{SCI} can have difficulty walking and can benefit from gait training \cite{Gorgey2018, Young2017, Lerner2017}.
Exoskeletons are suggested to give the patients assistance in gait training so that the patients can endure increased amount of training.
The assistance can be provided by an actuator, such as a motor, controlled by the exoskeletons control system.
In rehabilitation, the powered assistance from the actuator should ideally be enough to help the patient finish the motion correctly but not so much that the patient does not need to do anything.
This balance needs to come from the control system and calls for a smooth human-machine interaction.

\Ac{EMG} signal from muscle activity is non-linearly related to joint torque and can thus be used as a responsive human-machine control strategy for the actuator \cite{Young2017}.
Therefore, by measuring \ac{EMG} from the leg, a model could be constructed to estimate the \ac{KJM} from the muscle activity.
Such \ac{EMG}-based models have been developed, such as the modified Hill-type muscle model discussed in section \ref{sec:A-EMGBasedModels} \cite{Pizzolato2015, Erdemir2007, Lloyd2003}.
However, due to the nature of how \ac{EMG} is related to muscle forces, these models require complicated measurements such as maximum isometric muscle force, and muscle geometries.
A possible alternative to these kind of models is to train a \ac{NN} to map the \ac{EMG} signal as closely as possible to the joint torque moments.
A \ac{NN} is a machine learning model which is capable of representing a mapping function from one set of parameters to another. 
The mapping function could be trained to represent the mapping from \ac{EMG} to joint torque as closely as possible, based on an objective/error function.


% Kinematic data 

% Exoskeletons are .....

% What to talk about:
% Talk about exoskeletons:
% \begin{itemize}
%     \item Briefly how they assist with motion
%     \item Human-robot interaction
%     \item Smoothness of real-time control
%     \item Myoelectric control, parametric (Hill-type) vs. non-parametric (neural networks) \cite{Anam2012}.
% \end{itemize}

% Discuss limitations for using real-time analysis, e.g. data acquisition, complexity of analysis and processing power of microcontrollers on the exoskeleton.

% Discuss ANN together with EMG as an alternative to motion capture and complex analysis such as inverse dynamics and Hill-type muscle models. 
% Also briefly mention what ANN is all about and its limitations, specially in relation to overfitting and underfitting.



%We use the \emph{biblatex} package to handle our references.  We therefore use the
%command \texttt{parencite} to get a reference in parenthesis, like this
%\parencite[12]{Pizzolato2015}\cite{Pizzolato2015}.  It is also possible to include the author
%as part of the sentence using \texttt{textcite}, like talking about
%the work of \textcite{Lloyd2003}.

\section{Research Question}
Is \ac{ANN} feasible to use as a myoelectric control model in an exoskeleton and how does it perform when compared to joint moment prediction from inverse dynamics?
Is it possible to generalize \ac{EMG} to torque prediction across different conditions of the same subject or even across different subjects doing the same task?

\end{document}