\section{State of the Art}
This chapter first provides the state of the art of exoskeletons as assistive devices and how the estimation of joint moments can be used. 
Secondly, theoretical background of the different methods used for kinetics of human motion are discussed and they compared to each other.
This includes first a discussion about the links from EMG measurements to muscle activation and muscle synergies and to muscle forces resulting in joint moments and motion.
Thirdly, the chapter details the use of machine learning algorithms as an alternative to the kinetics methods and goes into the state of the art of different algorithms.

The ability to predict muscle forces and joint moments using measurements from the body is important to create smooth human-machine interface between a user and an assistive device.
Extensive research has been done towards gait modelling and joint moment prediction. 
Motion labs and inverse dynamics can be used to calculate joint angles and moments from motion and force plate data. 
The cause of the motion and moments, namely the muscle force, is however not well predicted here. 
Static and dynamic optimization methods have been used to predict the muscle forces but they lack the consideration of variations in muscle activation patterns between tasks and individuals \cite{Pizzolato2015, Sartori2012a}. 
The challenge of accounting for different muscle synergies for the same motion could be answered using EMG-driven models \cite{Pizzolato2015, Sartori2012a}.

\subsection{Exoskeletons as Assistive Devices}
\label{sec:A-Exoskeletons}
The exoskeleton, as it will be discussed here, refers to a mobile electromechanical structure which can be worn by a human to provide power amplification, support and/or assistance in human motion. Exoskeletons can also be used for haptic interactions \cite{Anam2012}, but the focus here will be on the groups mentioned before. More specifically, the focus is on exoskeletons as assistive devices for those with physical impairment, i.e. for rehabilitation and/or normal human motion. This excludes exoskeletons as power amplification for professional or military purposes.

To be able to provide assist to motion the exoskeleton needs a human-exoskeleton interaction together with a control system. 
The control system is how the human motion is translated into mechanical activation, and vice versa. 
\citeauthorint{Anam2012} list four different categorizations of control systems in exoskeletons, namely \textit{Model-based}, \textit{Hierarchy based}, \textit{Physical parameters based}, and \textit{Usage based}. 
For the purpose of this review the focus is set on the first categorization, i.e. \textit{Model-based} control systems.
The article states that the control strategy of an exoskeleton can be divided into two types depending on the model used, namely the \textit{dynamic model} and the \textit{muscle model} based control. 
One could argue for a third type being a brain model based control like the one implemented in \cite{Wang2017a}, however this type of model will not be discussed further here.

\citeauthor{Anam2012} divide the \textit{dynamic model} into three different categories, each with its strengths and weaknesses.
These models use systems physical characteristics and motion data to estimate moments and/or forces required from the exoskeleton. 
For example, one of the categories is the artificial intelligent method where one study has used the joint angle, angle velocity, and angle acceleration as input to the model and the joint torque as output, behaving like an inverse dynamics model \cite{Anam2012}.
%One might argue that what these models lack in relation to the exoskeleton is responsiveness. 

The \textit{muscle model} uses muscle neural activity and/or joint kinematics to predict muscle forces and resulting joint moments. These models can be split into the two categories: parametric and non-parametric muscle model. 
The difference between them is that in addition to EMG data the parametric model needs anthropometric data with information of muscle and joint dynamic \cite{Anam2012}.
The \textit{muscle model}, often called EMG-driven modelling, has been researched as a way to provide real-time prediction of muscle forces and joint torques \cite{Anam2012, durandau, Pizzolato2015}. 
Furthermore, as mentioned by \citeauthorint{Anam2012}, much effort has been made towards predicting the user's intentions beforehand by using EMG data.
The Hill-type muscle model is possibly the most researched parametric muscle model and some modifications of it have been made throughout the years to make it more accurate \cite{Lloyd2003, Anam2012, Pizzolato2015, Lee14-1}.
For the non-parametric muscle model an artificial neural network could be used \cite{Kiguchi2012,Lee14-1}.  

Chapter \ref{sec:A-MSModels} will compare and go into detail about the different parametric and non-parametric muscle models as well as some dynamic models and inverse dynamics.
To begin with, next chapter will briefly introduce the theoretical science between \ac{EMG} signals and their relationship (or non-relationship) to muscle activation and muscle forces.

\subsection{Electromyography and Muscle Activity}
\label{sec:A-EMG}
Muscle forces of the skeletal muscles are the forces produced by the human body to move or stabilize the human body. 
The muscle force is generated by muscle contraction which, without going into much detail, is activated by electrical \ac{AP} travelling through the nerves of the body to and through the muscle fibers. 
The \ac{AP} going through the muscle fibers and making the muscle contract is what is measured in \ac{EMG} recordings.
To put this more clearly, \ac{EMG} is potentially measuring muscle activation but it does not measure muscle force as that feature depends on so many other factors \cite{Enoka2016}.

There are two common methods of measuring EMG, one being the invasive method where a needle electrode is inserted percutaneously to the target muscle, and the second being the surface EMG method where electrodes are placed on the skin close to the target muscle. 
Although the invasive method can give more accurate signal, the \ac{sEMG} method is more widely used due to the fact that it is non-invasive and easier to implement. 
The continued discussion here about \ac{EMG}s will be focused on the \ac{sEMG} method.
When using the \ac{sEMG} method the source of the signal and the recording electrodes are separated by biological tissues (e.g. skin, fat, blood vessels).
The signal can still be detected as the \ac{AP} in the muscle fibers generates an electric field in the surrounding tissue with conducting properties.
However, these tissues act as spatial and temporal low-pass filters with resulting deforming effect on the signal.
A common method to partially compensate for the spatial low-pass filtering is to use a pair of electrodes positioned few cm apart.
This method also allows removing the common-mode component (e.g. $50Hz$ from power line interference) \cite{Farina2016}.

Other artifacts of the \ac{sEMG} signal include end-of-fiber effect, electrode-skin impedance

Due to all these implications and artifacts the signal recorded by the electrodes depends on a number of anatomical, physical, and detection system parameters. The following list are the most important factors as listed by \citeauthorpage{Farina2016}{41}:
\begin{quote}
(1) the thickness of the subcutaneous tissue layer (only for surface recordings),
(2) the depth of the sources within the muscle (for surface recordings) and the distance from the source to the electrodes (for intramuscular recordings),
(3) the inclination of the detection system with respect to the muscle fiber orientation (mainly for surface recordings),
(4) the length of the fibers (mainly for surface recordings),
(5) the location of the electrodes over the muscle (or within the muscle in case of intramuscular recordings),
(6) the spatial filter (electrode montage) used for signal detection, including the inter-electrode distance,
(7) the electrode size and shape(for surface recordings), and
(8) crosstalk among nearby muscles (for surface recordings).
\end{quote}
Thus, it is easy to see that the \ac{sEMG} can vary highly between individuals as well as between days and measurements of the same individual.

Guidelines have been made by \ac{SENIAM} in effort to make \ac{sEMG} procedures more standardized and comparable and to increase reproducibility of the experiment. 
The \ac{SENIAM} project is a European concerted action towards development of recommendations on sensors, sensor placement, signal processing, and modelling for \ac{sEMG} \cite{Hermens2000}.

Talk about filtering

Talk about muscle synergies

\subsection{Musculoskeletal models}
\label{sec:A-MSModels}
As previously stated, the \textit{Model-based} control of exoskeletons can be divided into either \textit{dynamic models} or \textit{muscle model} \cite{Anam2012}. 
The objective of both groups is essentially to estimate muscle forces and/or resulting joint moments and both can perhaps be considered as a subcategory of musculoskeletal (MS) models. 
However, what distinguishes these groups is whether the model uses data collected from the muscle activity or not, i.e. whether \ac{EMG} data is used or not.



\subsubsection{Hill-model}
\label{sec:A-HillModel}
The Hill-model 

