\section{Theoretical Background}
In this chapter, a detailed description about background of the degree project is presented together with related work. Discuss what is found useful and what is less useful. Use valid arguments. 

Explain what and how prior work / prior research will be applied on or used in the degree project /work (described in this thesis). Explain why and what is not used in the degree project and give valid reasons for rejecting the work/research.

Use references!

\subsection{EMG Measurements and Modelling}

CEINMS comprises EMG-driven and EMG-informed algorithms that have been previously published and tested. 
It operates on dynamic skeletal models possessing any number of degrees of freedom and musculotendon units and can be calibrated to the individual to predict measured joint moments and EMG patterns \cite{Pizzolato2015}.

CEINMS which stands for Calibrated EMG-Informed NMS is a modelling toolbox which uses EMG-drive neuromusculoskeletal (NMS) models and enables prediction of different neural control solutions for the same musculoskeletal geometry and measured movements. 

Static and dynamic optimization are other methods to estimate muscle forces and joint moments but, unlike the EMG-drive models, they cannot account account for variations in muscle activation patterns between tasks and individuals.
The article states that EMG-driven methods are not as accessible as optimization methods, having limited public availability or focus on isometric tasks rather than dynamic tasks. 
CEINMS is intended to increase said accessibility. 
It covers neural control solutions from EMG-driven, to hybrids between EMG-driven and static optimization, to full static optimization.

CEINMS use involves the three steps of \textit{calibration}, \textit{execution}, and \textit{validation}. \textit{Calibration} is to refine the subject's neuromuscular parameters to reduce the error between experimental joint moments and joint moments predicted by the CEINMS EMG-driven open-loop mode. There are two sets of parameters which can be refined during the calibration. The first set is the \textit{musculotendon unit's activation dynamics} set and the second set is the \textit{musculotendon contraction dynamics} set. 

The first set (i.e. activation dynamics) characterizes the transformation of muscle excitation, $e(t)$, to muscle activation, $a(t)$. The transformation is derived from the neural activation, $u(t)$, from equation \ref{eq:neural-activation} and the muscle activation, $a(t)$, from either equation \ref{eq:muscle-activation-1} or \ref{eq:muscle-activation-2}.
\begin{align}
\label{eq:neural-activation}
    u(t) &= \alpha e(t-d) - (C_1 + C_2)u(t-1) - C_1 C_2 u(t-2) \\
\label{eq:muscle-activation-1}
    a(t) &= \frac{e^{Au(t)} - 1}{e^A - 1}\\
\label{eq:muscle-activation-2}
    a(t) &= \begin{cases}
        \alpha^{act}\ln \left( \beta^{act}u(t)+1 \right) &, 0\leq u(t) < u_0\\
        mu(t) + c &, u_0 \leq u(t) \leq 1
    \end{cases}
\end{align}

The parameters to be refined in equation \ref{eq:neural-activation} are $C_1$ and $C_2$ which are the recursive coefficients. The muscle gain coefficient, $\alpha$, can be determined from these coefficients as:
\begin{align}
\label{eq:muscle-gain-coeff}
    \alpha - C_1 - C_2 - C_1 C_2 = 1 \text{ where } \left|C_1\right|,\left|C_2\right| < 1
\end{align}
and finally $d$ is the electromechanical delay.

The parameters to be refined in equations \ref{eq:muscle-activation-1} and \ref{eq:muscle-activation-2} is only the shape factor $A$ but $\alpha^{act}, \beta^{act}, m, c$ are only dependent on $A$. For equation \ref{eq:muscle-activation-1}: $A\in (-3, 0)$ but for equation \ref{eq:muscle-activation-2}: $A\in (0, 0.12]$.

The second set (i.e. contraction dynamics) relates to the force produced by the musculotendon unit (MTU) in contraction. The parameters to refine are the tendon slack length $l_{ts}$, the optimal fiber length $L_m^0$, and the muscle strength coefficient which is a multiplicative factor for the max isometric force $F^{max}$. The muscle strength coefficient can be shared by multiple muscles. 

CEINMS requires three setup files for \textit{calibration}, \textit{neural mapping}, and \textit{execution}.

\subsection{Related Work}
You should probably keep a heading about the related work here even though the entire chapter basically only contains related work.
