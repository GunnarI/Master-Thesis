\section{Theoretical Background}
In this chapter, a detailed description about background of the degree project is presented together with related work. Discuss what is found useful and what is less useful. Use valid arguments. 

Explain what and how prior work / prior research will be applied on or used in the degree project /work (described in this thesis). Explain why and what is not used in the degree project and give valid reasons for rejecting the work/research.

Use references!

\subsection{EMG Measurements and Modelling}
CEINMS comprises EMG-driven and EMG-informed algorithms that have been previously published and tested. 
It operates on dynamic skeletal models possessing any number of degrees of freedom and musculotendon units and can be calibrated to the individual to predict measured joint moments and EMG patterns.\cite{Pizzolato2015} 
CEINMS which stands for Calibrated EMG-Informed NMS is a modelling toolbox which uses EMG-drive neuromusculoskeletal (NMS) models and enables prediction of different neural control solutions for the same musculoskeletal geometry and measured movements. 
Static and dynamic optimization are other methods to estimate muscle forces and joint moments but, unlike the EMG-drive models, they cannot account account for variations in muscle activation patterns between tasks and individuals.
The article states that EMG-driven methods are not as accessible as optimization methods, having limited public availability or focus on isometric tasks rather than dynamic tasks.


\subsection{Related Work}
You should probably keep a heading about the related work here even though the entire chapter basically only contains related work.
