\titlepage

\begin{abstract}
Exoskeletons have potential to help with recovery during rehabilitation or assisting patients with impaired mobility.
An essential part of an active exoskeletons to pursue the role of assistance is its control system.
The exoskeletons needs to assist and reinforce motion of the user rather than hinder or counteract it.
One way of providing such a control system is to use signals from the body as an input to the system.
Factors such as motion of body parts, acceleration, velocity, rotation, reaction forces, and muscle activity can be measured to indicate the power needed at the joints.
\Ac{EMG} can be used to indicate muscle activity and is relatively easy to measure with low power consumption.
\Ac{EMG} signal is however non-linearly related to muscle force and joint moments.
Furthermore, the relationship depends on many factors which are hard to measure (e.g. muscle fiber length, elasticity, and maximum isometric muscle force).
In this thesis the possibility of using a \ac{NN} to map the \ac{EMG} signal to \ac{KJM} during gait is explored.
The aim is also to provide a basis for a model which could be used in real-time in an exoskeleton while using \ac{EMG} signal as the only input.
The \ac{KJM} to map the signal onto is estimated using a state-of-the-art motion capture system and modelling tools for human motion including \ac{IK} and \ac{ID} calculations.

\end{abstract}


\begin{otherlanguage}{swedish}
  \begin{abstract}
    %Träutensilierna i ett tryckeri äro ingalunda en oviktig faktor,
    %för trevnadens, ordningens och ekonomiens upprätthållande, och
    %dock är det icke sällan som sorgliga erfarenheter göras på grund
    %af det oförstånd med hvilket kaster, formbräden och regaler
    %tillverkas och försäljas Kaster som äro dåligt hopkomna och af
    %otillräckligt.
  \end{abstract}
\end{otherlanguage}


\tableofcontents

% Include Abbreviations list
\clearpage
\printacronyms[include-classes={abbrev, acro}, name=Abbreviations]
