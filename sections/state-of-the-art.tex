\section{Theoretical Background}
In this chapter, a detailed description about background of the degree project is presented together with related work. Discuss what is found useful and what is less useful. Use valid arguments. 

Explain what and how prior work / prior research will be applied on or used in the degree project /work (described in this thesis). Explain why and what is not used in the degree project and give valid reasons for rejecting the work/research.

Use references!

\subsection{EMG Measurements and Modelling}

\subsubsection{CEINMS}
CEINMS comprises EMG-driven and EMG-informed algorithms that have been previously published and tested. 
It operates on dynamic skeletal models possessing any number of degrees of freedom and musculotendon units and can be calibrated to the individual to predict measured joint moments and EMG patterns \cite{Pizzolato2015}.

CEINMS which stands for Calibrated EMG-Informed NMS is a modelling toolbox which uses EMG-drive neuromusculoskeletal (NMS) models and enables prediction of different neural control solutions for the same musculoskeletal geometry and measured movements. 

Static and dynamic optimization are other methods to estimate muscle forces and joint moments but, unlike the EMG-drive models, they cannot account account for variations in muscle activation patterns between tasks and individuals.
The article states that EMG-driven methods are not as accessible as optimization methods, having limited public availability or focus on isometric tasks rather than dynamic tasks. 
CEINMS is intended to increase said accessibility. 
It covers neural control solutions from EMG-driven, to hybrids between EMG-driven and static optimization, to full static optimization.

CEINMS use involves the three steps of \textit{calibration}, \textit{execution}, and \textit{validation}. \textit{Calibration} is to refine the subject's neuromuscular parameters to reduce the error between experimental joint moments and joint moments predicted by the CEINMS EMG-driven open-loop mode. There are two sets of parameters which can be refined during the calibration. The first set is the \textit{musculotendon unit's activation dynamics} set and the second set is the \textit{musculotendon contraction dynamics} set. 

The first set (i.e. activation dynamics) characterizes the transformation of muscle excitation, $e(t)$, to muscle activation, $a(t)$. The transformation is derived from the neural activation, $u(t)$, from equation \ref{eq:neural-activation} and the muscle activation, $a(t)$, from either equation \ref{eq:activation-exponential} or \ref{eq:activation-piecewise}.
\begin{align}
\label{eq:neural-activation}
    u(t) &= \alpha e(t-d) - (C_1 + C_2)u(t-1) - C_1 C_2 u(t-2) \\
\label{eq:activation-exponential}
    a(t) &= \frac{e^{Au(t)} - 1}{e^A - 1}\\
\label{eq:activation-piecewise}
    a(t) &= \begin{cases}
        \alpha^{act}\ln \left( \beta^{act}u(t)+1 \right) &, 0\leq u(t) < u_0\\
        mu(t) + c &, u_0 \leq u(t) \leq 1
    \end{cases}
\end{align}

The parameters to be refined in equation \ref{eq:neural-activation} are $C_1$ and $C_2$ which are the recursive coefficients. The muscle gain coefficient, $\alpha$, can be determined from these coefficients as:
\begin{align}
\label{eq:muscle-gain-coeff}
    \alpha - C_1 - C_2 - C_1 C_2 = 1 \text{ where } \left|C_1\right|,\left|C_2\right| < 1
\end{align}
and finally $d$ is the electromechanical delay.

The parameters to be refined in equations \ref{eq:activation-exponential} and \ref{eq:activation-piecewise} is only the shape factor $A$ but $\alpha^{act}, \beta^{act}, m, c$ are determined from $A$. 
For equation \ref{eq:activation-exponential}: $A\in (-3, 0)$ but for equation \ref{eq:activation-piecewise}: $A\in (0, 0.12]$. 
Note that equation \ref{eq:activation-exponential} and \ref{eq:activation-piecewise} are two different solutions for relation between neural and muscle activation. 
One can choose between them by defining the <activation> parameter as either <exponential\textbackslash{}> for equation \ref{eq:activation-exponential} or <piecewise\textbackslash{}> for equation \ref{eq:activation-piecewise}.

The second set (i.e. contraction dynamics) relates to the force produced by the musculotendon unit (MTU) in contraction. The parameters to refine are the tendon slack length $l_{ts}$, the optimal fiber length $L_m^0$, and the muscle strength coefficient which is a multiplicative factor for the max isometric force $F^{max}$. The muscle strength coefficient can be shared by multiple muscles. 

CEINMS requires three setup files for \textit{calibration}, \textit{neural mapping}, and \textit{execution}.

\subsubsection{EMG-Driven Forward-Dynamic Estimation of Muscle Force and Joint Moment about Multiple Degrees of Freedom in the Human Lower Extremity}

The article mentions the challenge due to the fact that prescribed joint moment and motion in the human body can be the result of different MTU excitation strategies. 
Optimization-driven methodologies have been used to study this behavior, where "MTUs are assumed to contribute to the experimentally measured joint moments according to a chosen criterion that is presumed to be generalizable across subjects and motor tasks". 
"However, it has been shown that in humans the neuromuscular redundancy is solved by means of the neural drive to MTUs, or MTU excitation. In this scenario, MTUs are recruited independently of the final joint moment and position, but rather based on the motor task to be performed, and on the personal history of training and pathology." \cite{Sartori2012a}

It is stated that EMG-driven modelling requires calibration of parameters that vary non-linearly between individuals due to anatomical and physiological differences.

"However, even though single-DOF models account for
the neuromuscular redundancy based on experimental EMG data,
it has never been examined whether they can be applied to predict
the MTU dynamics and the resulting joint moment with respect to
a different DOF than that used for calibration. In other words, is
the force, generated by the same MTU and driven by the same
input data during the same movement, predicted differently if
different single-DOF models are used?" \cite[p. 2]{Sartori2012a}
This means that they want to look at if predicted MTU force solution is generalizable across DOFs, i.e. if the prediction still holds when modelling with different DOFs or multiple DOFs. For this reason, they argue that the development of multi-DOF EMG-driven modelling is crucial to satisfy joint moments with respect to multiple DOFs simultaneously. 

The aim is to create a multi-DOF EMG-driven modelling for the lower extremities using the four DOFs: hip adduction-abduction (HipAA), hip flexion-extension (HipFE), knee flexion-extension (KneeFE), and ankle plantar-dorsi flexion (AnkleFE).
The data trial was conducted in a motion lab and motion data from both static anatomical poses and dynamic gait trials were collected. The dynamic trial included four motor tasks, namely walking, running, sidestepping, and crossover. Both calibration dataset and validation dataset were created from two separate trials included the four previously mentioned motor tasks. The calibration was to adjust parameters that vary non-linearly across subjects (such as tendon slack length and optimal fiber length) to minimize a certain objective function: $f_E = \left( E_{HipAA} + E_{HipFE} + E_{KneeFE} + E_{AnkleFE} \right)$

The article justifies the chose of tasks as they (1) allow producing substantially high moments about the four considered DOFs and (2) reflected different MTU recruitment strategies and contraction dynamics.

The multi-DOF EMG-driven model developed splits into five main components, namely Musculotendon Kinematics, Musculotendon Activation, Musculotendon Dynamics, Moment Computation, and Model Calibration. The setup is nicely illustrated in figure 1 in the article \cite[Fig. 1]{Sartori2012a}.

The validation procedure consisted of four tests:
\begin{enumerate}
    \item The first test assessed whether single-DOF models produced different force estimates for the same MTU using the same input data and during the same movement. The result revealed substantial differences with NRMSD (Normalized Root Mean Square Deviation) mostly around 0.5 for most muscles \cite[Fig. 3]{Sartori2012a}.
    \item The second test compared the joint moment prediction accuracy of the multi-DOF model to that of the four single-DOF models using the task-specific ensemble average curves. The multi-DOF model predicted joint moments simultaneously produced about four DOFs during the four considered motor tasks. The result showed that the multi-DOF model predicted moments, at each included DOF, with comparable performance to the four single-DOF models \cite[Fig. 4]{Sartori2012a}.
    \item The third test compared the multi-DOF model $F^{mt}$ (i.e. the MTU force) solutions to those obtained by the single-DOF models. See results on page 6 and figure 5.
    \item In the fourth test the multi-DOF model calibration and execution time were examined. See results on page 7 end of result chapter.
\end{enumerate}

"Experimental results also showed that both the single-DOF and multi-DOF EMG-driven models could not produce joint moment estimates that exactly matched the experimental joint moments (Figure 4). This is in part related to two main limitations of surface EMG: 1) the inability to access EMG data from deeply located MTUs, and 2) difficulties in characterizing the EMG frequency bandwidth to best drive the musculoskeletal model [35]." - \cite[p. 11]{Sartori2012a}

\subsubsection{Modeling the Human Knee for Assistive Technologies}

The article aims at developing an EMG-driven model which is fast enough to be used on an embedded system in assistive technologies to predict muscle forces and joint moments.

They propose a MTU model based on infinitely stiff tendons assumption and argue that this model is as accurate as "the sophisticated state-of-the-art models" \cite[p. 2]{Sartori2012} but with much higher computational speeds. Furthermore, they justify this assumption by pointing out that the tendon is normally very stiff with a strain of $3.3\%$ at maximum isometric force (MIF). They identify the sophisticated state-of-the-art MTU model as the elastic-tendon Hill-type muscle component (EHM) and their proposed new model as the stiff-tendon Hill-type muscle (SHM) component. In EHM the fiber length was determined at each time frame with a time consuming integration process. They argue that their SHM reduces this time since the maximum MTU force ($F^{mt}$) is then dependent only on fiber length ($l^m$) and not also the tendon length ($l^t$) which is now fixed to its slack length ($l^t_s$).

The procedure was to collect motion data from numerous motion lab trials, including static and dynamic trials (i.e. walking and running).
Six healthy male subjects were used and a 16 channel EMG system.
OpenSim was then used to apply inverse kinematics and inverse dynamics algorithms and get resulting estimated joint moments from the motion lab data.

The EMG-driven model consists of five main components, as explained by them, namely: muscle activation, musculotendon kinematics (MTK), musculotendon dynamics, moment computation, and model calibration \cite[p. 4]{Sartori2012}.

In their model they included 13 MTUs crossing the knee joint: semimembranosus (sm), semitendinosus (st), biceps femoris long head (bflh), biceps femoris short head (bfsh), sartorius (sar), tensor fascia latae (tfl), gracilis (gra), vastus lateralis (vl), vastus medialis (vm), vastus intermedius (vi), rectus femoris (rf), medial gastrocnemius (mg), and lateral gastrocnemius (lg).

In the results they show that not much difference is between muscle force prediction using the SHM compared to the EHM \cite[Fig. 3]{Sartori2012}. Furthermore, KFE moments predicted by both models were also shown to be similar. They also compare the SHM and EHM KFE moment prediction to the experimental result, i.e. $M_{KFE}$, which shows SHM prediction to be better correlated to $M_{KFE}$ \cite[Fig. 4]{Sartori2012} with $R^2 = 0.89$ and $RMSE = 7.89Nm$ between SHM KFE moments and $M_{KFE}$.

They mention that they ran the model on an embedded system with the SHM (Model(SHM,MTK)) or EHM (Model(EHM,MTK)) components. 
The Model(SHM,MTK) executed in $3.6 \pm 0.064 ms$, while the Model(EHM,MTK) needed $21.1 \pm 0.11 ms$ \cite[p. 8]{Sartori2012}.
However, they also point out that the embedded system still requires extra instrumentation to measure EMG and joint/segmental angles and moments. The data was gathered and loaded on the embedded system internal memory before running the model so this is not running in real-time as would probably be needed in an assistive device.

\subsubsection{Evaluation of different analytical methods for subject-specific scaling of musculotendon parameters}

\subsubsection{Performances of Hill-Type and Neural Network Muscle
Models—Toward a Myosignal-Based Exoskeleton}

"The aim of the study was to compare the performances of two types of muscle models in terms of predicting the moments developed at the human elbow joint complex based on joint kinematics and neuromuscular activity. The performance evaluation of the muscle models was required to implement them in a powered myosignal-driven exoskeleton (orthotic device). The experimental setup included a passive exoskeleton capable of measuring the joint kinematics and dynamics in addition to the muscle myosignal activity (EMG). Two types of models were developed and analyzed: (i) a Hill-based model and (ii) a neural network." \cite[p. 1]{Rosen1999}

\subsubsection{Surface Electromyography Methods}
Electromyography can be used to measure muscle activity. 
It is the measure of the electric potential generated in muscle cells after an action potential from motor neurons. 
The electric potential generated in the muscle can be measured via the non-invasive surface electrodes or invasive electrodes. 
When using surface electrodes the method is called surface electromyography (sEMG). 
This method can capture activity in a muscle rather accurately but it is not as easy to say what this activity represents. 
Muscle force cannot be directly obtained from EMG and furthermore EMG can be different between days and conditions of the subject. 
The tissue (e.g. skin- and fat-tissue) that is between the sEMG electrodes and the origin of the electric potential affects the signal. 
It acts as a filter with varying filtering characteristics depending on the position of electrodes, skin preparation, ambient temperature and electrical impedance \cite{Lloyd2003}.  
As pointed out by \citeauthor{Lloyd2003} \cite{Lloyd2003} different EMG activation patterns are observed between different subjects or different control tasks even though the joint angle and torque is the same.

"Different methods have been developed to estimate muscle forces. These methods include static and dynamic optimisation, which involve the use of inverse dynamics to track external joint moments and/or joint kinematics and estimation of muscle activations and forces to satisfy pre-selected objective criteria. However, optimisation methods cannot account for variations in muscle activation patterns  between tasks and individuals, even when joint angles and moments are the same." \cite{Pizzolato2015}

Some EMG-driven NMS models have been created to estimate muscle force and consequently joint torque from measured EMG together with joint angles. 
Some examples are the models described by  \citeauthor{Lloyd2003} \cite{Lloyd2003}, \citeauthor{Sartori2014} \cite{Sartori2014}, and \citeauthor{Sartori2012a} \cite{Sartori2012a} and used by CEINMS \cite{Pizzolato2015}. 
These models vary between completely EMG-driven models and hybrids between EMG-driven and static optimization.
The drawbacks of the completely EMG-drive model is that the experimental moments are not always well predicted.
This is due to limitations of sEMG, e.g. cross-talk, movement artifacts, the inability to access deep muscles, varying filtering characteristics of isolating tissue \cite{Lloyd2003, Sartori2014}.
Another consideration is the fact that the EMG is normalized to peak values to represent the percentage excitation level. 
This increases unreliability because the true maximum of EMG from any muscle is very difficult to attain \cite{Sartori2014}.
By combining static optimization with the EMG-driven model the limitations of both EMG-driven model and the static optimization model can be overcome \cite{Sartori2014}.

"it has been shown that, for the same joint moments and angles, different
individuals use different excitation patterns depending on the control
tasks, pathology, and training." - \cite[p. 3614]{Sartori2014}

\subsubsection{Neural Networks Electromyography Pattern Recognition}

Artificial Neural Networks have been suggested as an alternative to the Hill-based EMG-driven NMS model \cite{Rosen1999, Sepulveda1993}.
The model introduced by \citeauthor{Sepulveda1993} \cite{Sepulveda1993}

\subsection{Related Work}
You should probably keep a heading about the related work here even though the entire chapter basically only contains related work.
