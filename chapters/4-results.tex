\documentclass[../main.tex]{subfiles}

\begin{document}

\chapter{Results}
\label{sec:results}
The results can be split into two parts: (1) \textit{pre-training} which includes the experimental data after filtering and pre-processing and before the training of the \ac{NN}; and (2) \textit{post-training} which includes the output from the trained \ac{NN} model, including training and performance evaluation.

\section{Pre-training results}
\label{sec:results_pre-training-results}
Before training the \ac{NN} it is useful to examine what the data looks like.
Figure \ref{fig:kjm-average} shows a representation of the \ac{KJM} from all the gait cycles for subjects 1, 2 and 6.
Same plots can be found for subject 3 and 5 in the appendix \ref{sec:appendix}. 
Figure \ref{fig:emg-average} shows the corresponding \ac{EMG} activity of each muscle, grouped by the muscle's main function as per table \ref{tab:muscle-names}.
\begin{figure}[ht!]
     \centering
     \begin{subfigure}[b]{0.328\textwidth}
         \centering
         \includegraphics[width=\textwidth]{img/results/moment_avg/subject01_all_set1_moment_avg_w_minmax.png}
     \end{subfigure}
     \hfill
     \begin{subfigure}[b]{0.328\textwidth}
         \centering
         \includegraphics[width=\textwidth]{img/results/moment_avg/subject02_all_set1_moment_avg_w_minmax.png}
     \end{subfigure}
     \hfill
     \begin{subfigure}[b]{0.328\textwidth}
         \centering
         \includegraphics[width=\textwidth]{img/results/moment_avg/subject06_all_set1_moment_avg_w_minmax.png}
     \end{subfigure}
    \caption{Normalized \ac{KJM} for subjects 1, 2, and 6, from left to right respectively. The x-axis is the percentage of a single gait cycle. The solid blue line is the average from all the gait cycles and the shaded area around it is the standard deviation.}
    \label{fig:kjm-average}
\end{figure}
\begin{figure}[ht!]
    \centering
    \includegraphics[width=\textwidth]{img/results/emg_avg/all_subject_grid_emg_avg.png}
    \caption{Filtered and normalized \ac{EMG} from subjects 1, 2, and 6. The x-axis is the percentage of a single gait cycle. The solid blue lines represent the average from all the gait cycles for each muscle and the shaded area with corresponding color is the standard deviation.}
    \label{fig:emg-average}
\end{figure}
% \begin{figure}
%     \centering
%     \includegraphics[width=0.75\textwidth]{img/results/20190429_06_moments_normal_walk.png}
%     \caption{Knee joint moments from part of the trial of subject00 where gait cycles are considered as "normal" walking speed. Positive moment means knee flexion and negative knee extension. The values are normalized as explained in the chapter \ref{sec:methods}. The light-blue area around the average represents the standard deviation. This dataset includes 523 gait cycles. See figure \ref{fig:knee-joint-moments-stat-compl} for complementing dataset, i.e. "slow" and "fast" walking speeds. }
%     \label{fig:knee-joint-moments-stat}
% \end{figure}
The figures show the subjects have similar patterns for \acp{KJM} and \ac{EMG} signals over the gait cycles.
Note that the pattern for subject 1 is not as clear as for subjects 2 and 6 when looking at it on the same scale.
However, as indicated by the ``fastest'' and ``slowest'' cycles, it is because subject 1 generally walked slower, in most trials, compared to their maximum speed.
The similarities in the patterns indicates that an \ac{NN} might be able to generalize between subjects, at least up to some degree.
Nevertheless, there are also some dissimilarities between the signals which are discussed further in chapter \ref{sec:discussion}.

Another notable observation is the amplitude of the normalized \ac{EMG}.
It indicates that most of the values have below $0.5$ amplitude.
This suggests that the normalization method might not be adequate.
The discrepancy between the signals from gluteus maximus (GlutMax) may suggest the same. 
This is further discussed in section \ref{sec:discussion}.

Figures x - x show the mixed datasets from cases 1-5 as described in the methods section.

\section{Post-training results}
\label{sec:post-training-results}


% From figure \ref{fig:emg-to-moment-representation} it is clear that the \ac{EMG} amplitude does not always line up with the knee moment and furthermore that when looking at each specific time step, similar \ac{EMG} values can correspond to very different moment values. 
% This can be easily identified at time stamps $0.38s$ and $0.54$ presented with the horizontal lines cutting the moment plot at very different values but the \ac{EMG} at similar values, no matter the muscle.
% For this reason it may be unsuccessful to train a regular \ac{MLP} which bases its prediction on the input at each time point exclusively.
% This is where the stateful \ac{LSTM} model comes in where the outputs from previous predictions are also included as input to the current prediction. 
% This way the model may use the previous moment values as addition feature to the \ac{EMG} input.
% Thus, at time stamps $0.38s$ and $0.54$ the model may know if previous moment value was flexion or extension and base the prediction on that knowledge.

% \begin{figure}
%      \centering
%      \begin{subfigure}[b]{0.325\textwidth}
%          \centering
%          \includegraphics[width=\textwidth]{img/results/emg_avg/subject01_all_set1_knee_extensors_emg_avg.png}
%      \end{subfigure}
%      \hfill
%      \begin{subfigure}[b]{0.325\textwidth}
%          \centering
%          \includegraphics[width=\textwidth]{img/results/emg_avg/subject02_all_set1_knee_extensors_emg_avg.png}
%      \end{subfigure}
%      \hfill
%      \begin{subfigure}[b]{0.325\textwidth}
%          \centering
%          \includegraphics[width=\textwidth]{img/results/emg_avg/subject06_all_set1_knee_extensors_emg_avg.png}
%      \end{subfigure}
%      \hfill
%      \begin{subfigure}[b]{0.325\textwidth}
%          \centering
%          \includegraphics[width=\textwidth]{img/results/emg_avg/subject01_all_set1_knee_flexors_emg_avg.png}
%      \end{subfigure}
%      \hfill
%      \begin{subfigure}[b]{0.325\textwidth}
%          \centering
%          \includegraphics[width=\textwidth]{img/results/emg_avg/subject02_all_set1_knee_flexors_emg_avg.png}
%      \end{subfigure}
%      \hfill
%      \begin{subfigure}[b]{0.325\textwidth}
%          \centering
%          \includegraphics[width=\textwidth]{img/results/emg_avg/subject06_all_set1_knee_flexors_emg_avg.png}
%      \end{subfigure}
%      \hfill
%      \begin{subfigure}[b]{0.325\textwidth}
%          \centering
%          \includegraphics[width=\textwidth]{img/results/emg_avg/subject01_all_set1_ankle_flexors_emg_avg.png}
%      \end{subfigure}
%      \hfill
%      \begin{subfigure}[b]{0.325\textwidth}
%          \centering
%          \includegraphics[width=\textwidth]{img/results/emg_avg/subject02_all_set1_ankle_flexors_emg_avg.png}
%      \end{subfigure}
%      \hfill
%      \begin{subfigure}[b]{0.325\textwidth}
%          \centering
%          \includegraphics[width=\textwidth]{img/results/emg_avg/subject06_all_set1_ankle_flexors_emg_avg.png}
%      \end{subfigure}
%      \hfill
%      \begin{subfigure}[b]{0.325\textwidth}
%          \centering
%          \includegraphics[width=\textwidth]{img/results/emg_avg/subject01_all_set1_hip_extensor_emg_avg.png}
%      \end{subfigure}
%      \hfill
%      \begin{subfigure}[b]{0.325\textwidth}
%          \centering
%          \includegraphics[width=\textwidth]{img/results/emg_avg/subject02_all_set1_hip_extensor_emg_avg.png}
%      \end{subfigure}
%      \hfill
%      \begin{subfigure}[b]{0.325\textwidth}
%          \centering
%          \includegraphics[width=\textwidth]{img/results/emg_avg/subject06_all_set1_hip_extensor_emg_avg.png}
%      \end{subfigure}
%     \caption{Muscle activity in relation to knee joint moments. All activities and moments are taken as average from 523 different gait cycles considered as "normal" walking speeds for subject00.}
%     \label{fig:emg-to-moment-representation}
% \end{figure}
% \begin{figure}
%      \centering
%      \begin{subfigure}[b]{0.45\textwidth}
%          \centering
%          \includegraphics[width=\textwidth]{img/results/norm_emg_knee_ext_hline.png}
%      \end{subfigure}
%      \hfill
%      \begin{subfigure}[b]{0.45\textwidth}
%          \centering
%          \includegraphics[width=\textwidth]{img/results/norm_emg_knee_flex_hline.png}
%      \end{subfigure}
%      \hfill
%      \begin{subfigure}[b]{0.45\textwidth}
%          \centering
%          \includegraphics[width=\textwidth]{img/results/norm_emg_hip_ext_hline.png}
%      \end{subfigure}
%      \hfill
%      \begin{subfigure}[b]{0.45\textwidth}
%          \centering
%          \includegraphics[width=\textwidth]{img/results/norm_emg_ankle_flex_hline.png}
%      \end{subfigure}
%     \caption{Muscle activity in relation to knee joint moments. All activities and moments are taken as average from 523 different gait cycles considered as "normal" walking speeds for subject00.}
%     \label{fig:emg-to-moment-representation}
% \end{figure}

\end{document}
