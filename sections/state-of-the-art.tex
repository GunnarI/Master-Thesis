\section{State of the Art}
A typical exoskeleton would have a motor or some sort of actuator to produce torque at the joints (e.g. knee joint) to enable or assist the user.
The amount of torque produced for any given motion depends on the control system of the exoskeleton.
Many different methods have been researched and implemented towards the goal of making a good control system that provides exactly the assist that the user wants and/or needs.
The key to that goal is to estimate or predict the joint torque associated with a certain motion performed by the user.

This chapter discusses the state of the art of exoskeleton control systems and joint torque and muscle force prediction models.
Note that the overall focus is on lower extremity kinematics, joints, and muscles.
First, the discussion is aimed towards methods and models that are used today for kinetics of human motion analysis in general research and clinical applications.
This includes discussion about data acquisition, especially \ac{EMG} data, and its relation to the body dynamics, e.g. muscle forces.
Second, the discussion is narrowed towards the methods and models that can be used as control systems in an exoskeleton.
The flexibility for computation and data acquisition is lower in an exoskeleton and thus the number of methods that could be used more limited.
Third, the discussion is focused toward the state of the art of one particular method which consists of joint torque prediction using \ac{ANN} with \ac{EMG} data acquisition. 
This also includes general discussion about the research of different \acp{ANN} that have been used with \ac{EMG} data.

%This chapter first provides the state of the art of exoskeletons as assistive devices and how the estimation of joint moments can be used. Secondly, theoretical background of the different methods used for kinetics of human motion are discussed and they compared to each other. This includes first a discussion about the links from \ac{EMG} measurements to muscle activation and muscle synergies and to muscle forces resulting in joint moments and motion. Thirdly, the chapter details the use of machine learning algorithms as an alternative to the kinetics methods and goes into the state of the art of different algorithms.

%The ability to predict muscle forces and joint moments using measurements from the body is important to create smooth human-machine interface between a user and an assistive device. Extensive research has been done towards gait modelling and joint moment prediction. Motion labs and inverse dynamics can be used to calculate joint angles and moments from motion and force plate data. The cause of the motion and moments, namely the muscle force, is however not well predicted here. Static and dynamic optimization methods have been used to predict the muscle forces but they lack the consideration of variations in muscle activation patterns between tasks and individuals \cite{Pizzolato2015, Sartori2012a}. The challenge of accounting for different muscle synergies for the same motion could be answered using \ac{EMG}-driven models \cite{Pizzolato2015, Sartori2012a}.

\subsection{Kinetics of Human Motion: methods, models, and data acquisition}
\label{sec:A-MSModels}
Kinetics of human motion is the study of the forces that cause human motion. 
This includes muscle forces as well as external forces, such as \ac{GRF}.
It is challenging to measure \textit{in vivo} muscle forces and it requires at least a minimally invasive methods or an open surgery, so that a force transducer can be placed on a tendon attached to the muscle \cite{Erdemir2007}. 
Thus, in general clinical application as well as in exoskeletons, measuring muscle forces directly is hardly an option.
However, it is possible with non-invasive methods to measure reaction forces like \ac{GRF}, human motion (e.g. motion capture system), and \ac{EMG}.
Several models have then been developed to estimate joint torques and muscle forces using this data \cite{Erdemir2007}.
For the purpose of this chapter these methods will be split into two groups, namely those who use \ac{EMG} data (\ac{EMG}-based models) and those who do not use \ac{EMG} data (Non-\ac{EMG}-based models).
This chapter begins with the models that do not use \ac{EMG} data.
That includes inverse dynamics for joint torque estimations and static and dynamic optimization for muscle force estimation. 
Then it goes into detail about \ac{EMG} and its relation to muscle activity before discussing the models that do use \ac{EMG} data, namely \ac{EMG}-driven models.

%As previously stated, the \textit{Model-based} control of exoskeletons can be divided into either \textit{dynamic models} or \textit{muscle model} \cite{Anam2012}. 
%The objective of both groups is essentially to estimate muscle forces and/or resulting joint moments and both can perhaps be considered as a subcategory of musculoskeletal (MS) models. 
%However, what distinguishes these groups is whether the model uses data collected from the muscle activity or not, i.e. whether \ac{EMG} data is used or not.
\subsubsection{Non-\ac{EMG}-based Models}
\label{sec:A-NonEMGBasedModels}
\textit{Inverse dynamics} is the use of kinematics and reaction forces, e.g. motion capture, scaled model, and force plate data from a motion lab, to calculate joint torques estimations.
As the method works with the equation of motion, it relies on correct scaling of each body segment, and joints angles and positions relative to the reaction forces \cite{Buchanan2004,Erdemir2007}.
Extensive research has been done on inverse dynamics and it is widely used in clinical applications such as, for example, gait analysis in rehabilitation \cite{Erdemir2007,Buchanan2004,Pizzolato2015}.
OpenSim \cite{Delp2007, Seth2018} is an open-source software offering biomechanical models and simulation tools to, among many other things, model and run inverse dynamic solution on a scaled musculoskeletal model \cite{Buchanan2004, Delp2007, Seth2018, Pizzolato2017}.
\todo{Possibly talk more about the methods used in OpenSim}

The inverse dynamics method alone does not provide any information about muscle activation and muscle forces.
There are several methods that have been implemented to estimate muscle activation and muscle forces \cite{Erdemir2007}.
Perhaps the most notable ones that are not \ac{EMG}-drive are the static and dynamic optimization methods \cite{Erdemir2007, Delp2007, Pizzolato2015}. 

The \textit{static optimization} method uses optimization with an objective function such as, for example, minimizing the sum of muscle forces from all muscles. 
Through iteration the muscle forces, that both fit to the estimated joint torque from the inverse dynamics model and satisfy the objective function, are calculated \cite{Erdemir2007}. 
\todo{Write about pros and cons of this method and include more references}

\textit{Dynamic optimization} is an alternative to static optimization which, in short, uses forward dynamics to find the set of muscle activations that best fit the experimental kinematic data (e.g. with objective to minimize tracking error in kinematics and \ac{GRF}). 
This method does not require inverse dynamics to first estimate joint torques since the constraint is rather the measured motion and \ac{GRF} data.
\citeauthorint{Erdemir2007} point out that this method is beneficial as compared to the static optimization because of its more straightforward inclusion of muscle dynamics within the solution.
Furthermore, it is less vulnerable toward errors in the experimental measurements as the inclusion of the kinematic data is somewhat "weak", i.e. there is a slack on the kinematic data in favor of more intact muscle dynamics.
However, the dynamic optimization method is more computationally heavy when compared to the static optimization method \cite{Erdemir2007}.

As pointed out by \citeauthorint{Pizzolato2015} the optimization method, including static and dynamic optimization, cannot account for variations in muscle activation patterns (i.e. muscle synergies) between tasks and individuals.
For example, if the objective function is to minimize the sum of muscle forces that fit a certain joint moment, it does not account for the fact different muscle synergies can produce the same joint moment. 
Muscle synergies have been shown to be task-specific and vary between individuals and different pathologies \cite{Ivanenko2016, Safavynia2011}.
To account for this drawback, muscle activation need to be measured and accounted for, for example by using \ac{sEMG} and \ac{EMG}-driven models \cite{Pizzolato2015}.

\subsubsection{Electromyography and Muscle Activity}
\label{sec:A-EMG}
Muscle forces of the skeletal muscles are the forces produced by the human body to move or stabilize the human body. 
The muscle force is generated by muscle contraction which, without going into much detail, is activated by electrical \ac{AP} travelling through the nerves of the body to and through the muscle fibers. 
The \ac{AP} going through the muscle fibers and making the muscle contract is what is measured in \ac{EMG} recordings.
To put this more clearly, \ac{EMG} is potentially measuring muscle activation but it does not measure muscle force as that feature depends on so many other factors \cite{Enoka2016}.

There are two common methods of measuring \ac{EMG}, one being the invasive method where a needle electrode is inserted percutaneously to the target muscle, and the second being the surface \ac{EMG} method where electrodes are placed on the skin close to the target muscle. 
Although the invasive method can give more accurate signal, the \ac{sEMG} method is more widely used due to the fact that it is non-invasive and easier to implement. 
The continued discussion here about \ac{EMG}s will be focused on the \ac{sEMG} method.
When using the \ac{sEMG} method the source of the signal and the recording electrodes are separated by biological tissues (e.g. skin, fat, blood vessels).
The signal can still be detected as the \ac{AP} in the muscle fibers generates an electric field in the surrounding tissue with conducting properties.
However, these tissues act as spatial and temporal low-pass filters with resulting deforming effect on the signal.
A common method to partially compensate for the spatial low-pass filtering is to use a pair of electrodes positioned few cm apart.
This method also allows removing the common-mode component (e.g. $50Hz$ from power line interference) \cite{Farina2016}.

Some other artifacts of the \ac{sEMG} signal include, for example end-of-fiber effect, electrode-skin impedance, and crosstalk \cite{Farina2016}.
Due to all these implications and artifacts the signal recorded by the electrodes depends on a number of anatomical, physical, and detection system parameters. The following list are the most important factors as listed by \citeauthorpage{Farina2016}{41}:
\begin{quote}
(1) the thickness of the subcutaneous tissue layer (only for surface recordings),
(2) the depth of the sources within the muscle (for surface recordings) and the distance from the source to the electrodes (for intramuscular recordings),
(3) the inclination of the detection system with respect to the muscle fiber orientation (mainly for surface recordings),
(4) the length of the fibers (mainly for surface recordings),
(5) the location of the electrodes over the muscle (or within the muscle in case of intramuscular recordings),
(6) the spatial filter (electrode montage) used for signal detection, including the inter-electrode distance,
(7) the electrode size and shape(for surface recordings), and
(8) crosstalk among nearby muscles (for surface recordings).
\end{quote}
Thus, it is easy to see that the \ac{sEMG} can vary highly between individuals as well as between days and measurements of the same individual.

Guidelines have been made by \ac{SENIAM} in effort to make \ac{sEMG} procedures more standardized and comparable and to increase reproducibility of the experiment. 
The \ac{SENIAM} project is a European concerted action towards development of recommendations on sensors, sensor placement, signal processing, and modelling for \ac{sEMG} \cite{Hermens1999, Hermens2000}.
One obvious setback of \ac{sEMG}, as compared to using the invasive method, is that it only possible to measure muscles near the surface.
If the muscle is located behind other muscles or if too much tissue is between the source of the signal and the electrodes then the "correct" signal cannot be detected. 
Furthermore, the placement of the electrodes should not be too close to either the innervation zone or the end zone where the tendon is.
In fact it has been suggested that the best potition is right in between those zones \cite{Farina2016, Hermens2000}.
Also, the position of the electrodes cannot be to close to another muscle with the risk of crosstalk \cite{Farina2016, Hermens2000}.
The \ac{SENIAM} project has identified 14 muscles on lower extremities, including the hip, where \ac{sEMG} can be measured.
They have also provided guidelines on where to place the electrodes.
See table \ref{tab:SENIAM-mus-and-loc} for the list of these 14 muscles as well as short description of recommended electrodes locations \cite{Hermens1999}.

{
\renewcommand{\arraystretch}{1.5}
\begin{table}[ht!]
    \centering
    \caption{\ac{SENIAM}: Recommendations for sensor locations on lower extrimity muscles. Muscle names and location description are extracted from the \ac{SENIAM} website (\href{www.seniam.org}{www.seniam.org}). Visit the website or see \cite{Hermens1999} for more information.}
    \label{tab:SENIAM-mus-and-loc}
    \small
    \begin{tabular}{>{\raggedright}p{4.2cm}|p{8cm}}
        Muscles & Electrodes Location \\ \hline
        Gluteus (Maximus) & 50\% on the line between the sacral vertebrae and the greater trochanter. This position corresponds with the greatest prominence of the middle of the buttocks well above the visible bulge of the greater trochanter. \\
        Gluteus (Medius) & 50\% on the line from the crista iliaca to the trochanter.\\
        Tensor Fasciae Latae & On the line from the anterior spina iliaca superior to the lateral femoral condyle in the proximal 1/6.\\
        Quadriceps Femoris (rectus femoris) & The electrodes need to be placed at 50\% on the line from the anterior spina iliaca superior to the superior part of the patella\\
        Quadriceps Femoris (vastus medialis) & 80\% on the line between the anterior spina iliaca superior and the joint space in front of the anterior border of the medial ligament.\\
        Quadriceps Femoris (vastus lateralis) & 2/3 on the line from the anterior spina iliaca superior to the lateral side of the patella.\\
        Biceps Femoris (long and short head) & 50\% on the line between the ischial tuberosity and the lateral epicondyle of the tibia.\\
        Semitendinosus & 50\% on the line between the ischial tuberosity and the medial epycondyle of the tibia.\\
        Tibialis Anterior & 1/3 on the line between the tip of the fibula and the tip of the medial malleolus.\\
        Peroneus Longus & 25\% on the line between the tip of the head of the fibula to the tip of the lateral malleolus.\\
        Soleus & Anterior to the tendon of the m. peroneus longus at 25\% of the line from the tip of the lateral malleolus to the fibula-head.\\
        Gastrocnemius Medialis & On the most prominent bulge of the muscle.\\
        Gastrocnemius Lateralis & 1/3 of the line between the head of the fibula and the heel.
    \end{tabular}
\end{table}
}

\todo{Add short text here about the filtering and normalization of the EMG signal}

\clearpage
\subsubsection{\ac{EMG}-based Models}
\label{sec:A-EMGBasedModels}
As previously mentioned, \ac{EMG} can only measure muscle activation and not muscle forces.
\ac{EMG}-based models are models developed towards estimating muscle forces and/or joint torques base on \ac{EMG} data \cite{Pizzolato2015, Erdemir2007}.

Hill-Type models... \todo{\cite{Lloyd2003,WangMarc,Winby2008} and maybe \cite{Rosen1999} to talk about Hill-Type models}

While static and dynamic optimization are methods to estimate muscle forces from kinematic and reaction force data only, \ac{EMG}-driven models use also \ac{EMG} data \cite{Pizzolato2015}.
As previously stated, \citeauthorint{Pizzolato2015} recognize the drawback of the optimization methods in regards to synergy variation. 
Furthermore, they note that the optimization methods are easily accessible, which is for example reflected by the fact that OpenSim, a tool widely used by researchers, uses those methods in its models \cite{Delp2007,Pizzolato2015,Seth2018}. 
They further state that this accessibility does not apply for EMG-driven methods which "have been developed by different research groups and they are not publicly available, or are limited to isometric tasks" \cite[p. 1]{Pizzolato2015}.
\ac{CEINMS} is a tool developed to increase said availability and allows researchers to use \ac{EMG}-driven (or \ac{EMG}-informed) models on collected kinematic, reaction force, and \ac{EMG} data \cite{Pizzolato2015}.

%\citeauthorpage{Erdemir2007}{132} state that "actual estimates of muscle forces can only be obtained with computational models in which the skeleton and muscles are both represented".


\subsection{Exoskeletons as Assistive Devices}
\label{sec:A-Exoskeletons}
The exoskeleton, as it will be discussed here, refers to a mobile electromechanical structure which can be worn by a human to provide power amplification, support and/or assistance in human motion. Exoskeletons can also be used for haptic interactions \cite{Anam2012}, but the focus here will be on the groups mentioned before. More specifically, the focus is on exoskeletons as assistive devices for those with physical impairment, i.e. for rehabilitation and/or normal human motion. This excludes exoskeletons as power amplification for professional or military purposes.

To be able to provide assist to motion the exoskeleton needs a human-exoskeleton interaction together with a control system. 
The control system is how the human motion is translated into mechanical activation, and vice versa. 
\citeauthorint{Anam2012} list four different categorizations of control systems in exoskeletons, namely \textit{Model-based}, \textit{Hierarchy based}, \textit{Physical parameters based}, and \textit{Usage based}. 
For the purpose of this review the focus is set on the first categorization, i.e. \textit{Model-based} control systems.
The article states that the control strategy of an exoskeleton can be divided into two types depending on the model used, namely the \textit{dynamic model} and the \textit{muscle model} based control. 
%One could argue for a third type being a brain model based control like the one implemented in \cite{Wang2017a}, however this type of model will not be discussed further here.

\citeauthor{Anam2012} divide the \textit{dynamic model} into three different categories, each with its strengths and weaknesses.
These models use systems physical characteristics and motion data to estimate moments and/or forces required from the exoskeleton. 
For example, one of the categories is the artificial intelligent method where one study has used the joint angle, angle velocity, and angle acceleration as input to the model and the joint torque as output, behaving like an inverse dynamics model \cite{Anam2012}.
%One might argue that what these models lack in relation to the exoskeleton is responsiveness. 

The \textit{muscle model} uses muscle neural activity and/or joint kinematics to predict muscle forces and resulting joint moments. These models can be split into the two categories: parametric and non-parametric muscle model. 
The difference between them is that in addition to \ac{EMG} data the parametric model needs anthropometric data with information of muscle and joint dynamic \cite{Anam2012}.
This is much like the \ac{EMG}-driven models discussed in the previous chapter \cite{Pizzolato2015}.
The \textit{muscle model} has been researched as a way to provide real-time prediction of muscle forces and joint torques \cite{Anam2012, durandau, Pizzolato2015}. 
Furthermore, as mentioned by \citeauthorint{Anam2012}, much effort has been made towards predicting the user's intentions beforehand by using \ac{EMG} data.
The Hill-type muscle model is possibly the most researched parametric muscle model and some modifications of it have been made throughout the years to make it more accurate \cite{Lloyd2003, Anam2012, Pizzolato2015, Lee14-1}.
For the non-parametric muscle model an artificial neural network could be used \cite{Kiguchi2012,Lee14-1}.  
\todo{Find more references towards exoskeletons and do more comparisons between methods}

\subsection{Artificial neural networks in biomechanics}
This chapter has yet to be written\todo{Start with general discussion about ANN and its developments and recent advanced within biomechanics}
\\
\cite{Ardestani2014,Liu2009,Lee14-1,Smith2009,Naeem2012,Kiguchi2012}\todo{Do comparison of different ANN methods used with EMG data that have been researched.}
