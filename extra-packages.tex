% The TODO package
\usepackage[textwidth=33mm, textsize=tiny]{todonotes}

% Clickable links
\usepackage[hidelinks]{hyperref}

% For better vector notation arrow, command example: \vv{x}
\usepackage{esvect}

% The tikz package to create figures
\usepackage{tikz}
\usepackage{tikzscale}
%\usetikzlibrary{external}
%\tikzexternalize[prefix=tikz/]

% For subfigures
\usepackage{subcaption}

% For align equations
\usepackage{amsmath}

% For special table actions
\usepackage{array}

% To be able to compile each chapter separately 
\usepackage{subfiles}

% Setup for abbreviations
\usepackage{enumitem}
\newlength\myitemwidth
\setlength\myitemwidth{5em} % <<< choose what you need here
\newlist{myacronymlist}{description}{1}
\setlist[myacronymlist]{
  labelindent = 0pt ,
  labelsep    = 0pt ,
  leftmargin  = \myitemwidth ,
  labelwidth  = \myitemwidth ,
  itemindent  = 0pt 
  %format      = \normalfont
}
\usepackage{acro}
\DeclareAcroListStyle{myliststyle}{list}{
  list = myacronymlist
}
\acsetup{
    list-style  = myliststyle,
    list-caps   = true,
    first-style = default,
    single      = true
}
\DeclareAcroFirstStyle{InstAcronyms}{inline}{
    only-short  = true ,
    brackets    = false
}

\DeclareAcronym{EMG}{
    short   = EMG ,
    long    = electromyography ,
    class   = abbrev
}

\DeclareAcronym{ANN}{
    short           = ANN ,
    short-plural    = s ,
    long            = artificial neural network ,
    class           = abbrev
}

\DeclareAcronym{NN}{
    short           = NN ,
    short-plural    = s ,
    long            = neural network ,
    class           = abbrev
}

\DeclareAcronym{MLP}{
    short           = MLP ,
    short-plural    = s ,
    long            = multilayer perceptron ,
    class           = abbrev
}

\DeclareAcronym{RNN}{
    short           = RNN ,
    short-plural    = s ,
    long            = recurrent neural network ,
    class           = abbrev
}

\DeclareAcronym{DNN}{
    short           = DNN ,
    short-plural    = s ,
    long            = deep neural network ,
    class           = abbrev
}

\DeclareAcronym{LSTM}{
    short           = LSTM ,
    short-plural    = s ,
    long            = long short-term memory ,
    class           = abbrev
}

\DeclareAcronym{sEMG}{
    short   = sEMG ,
    long    = surface electromyography ,
    class   = abbrev
}

\DeclareAcronym{AP}{
    short   = AP ,
    long    = action potential ,
    class   = abbrev
}

\DeclareAcronym{SENIAM}{
    short       = SENIAM ,
    long        = Surface EMG for a Non-Invasive Assessment of Muscles ,
    first-style = InstAcronyms ,
    class       = acro
}

\DeclareAcronym{CEINMS}{
    short       = CEINMS ,
    long        = Calibrated EMG-Informed NMS Modelling Toolbox ,
    first-style = InstAcronyms ,
    class       = acro
}

\DeclareAcronym{MS}{
    short   = MS ,
    long    = musculoskeletal ,
    class   = abbrev
}

\DeclareAcronym{MU}{
    short   = MU ,
    long    = motor unit ,
    short-plural    = s ,
    class   = abbrev
}

\DeclareAcronym{MTU}{
    short   = MTU ,
    long    = musculotendon unit ,
    short-plural    = s ,
    class   = abbrev
}

\DeclareAcronym{NMS}{
    short   = NMS ,
    long    = neuromusculoskeletal ,
    class   = abbrev
}

\DeclareAcronym{IK}{
    short   = IK ,
    long    = inverse kinematics ,
    class   = abbrev
}

\DeclareAcronym{ID}{
    short   = ID ,
    long    = inverse dynamics ,
    class   = abbrev
}

\DeclareAcronym{GRF}{
    short           = GRF ,
    short-plural    = s ,
    long            = ground reaction force ,
    class           = abbrev ,
    single          = ground reaction force
}

\DeclareAcronym{FP}{
    short           = FP ,
    short-plural    = s ,
    long            = force plate ,
    class           = abbrev ,
    single          = force plate
}

\DeclareAcronym{HJC}{
    short           = HJC ,
    short-plural    = s ,
    long            = hip joint center ,
    class           = abbrev ,
    single          = hip joint center
}

\DeclareAcronym{KJC}{
    short           = KJC ,
    short-plural    = s ,
    long            = knee joint center ,
    class           = abbrev ,
    single          = knee joint center
}

\DeclareAcronym{PiG}{
    short           = PiG ,
    long            = Plug-in Gait ,
    class           = abbrev ,
}

\DeclareAcronym{CGM}{
    short           = CGM ,
    long            = Conventional Gait Model ,
    class           = abbrev
}


% Command to reference author in text and display ref number
%\newcommand\citeauthorint[1]{\citeauthor{#1} \cite{#1}}
\newcommand\citeauthorpage[2]{\citeauthor{#1} \parencite[#2]{#1}}

% This is just to get some nonsense text in this template, can be safely removed
\usepackage{blindtext}