\documentclass[../main.tex]{subfiles}

\begin{document}

\chapter{Methods}
The process from collecting data to training the \ac{NN} can be divided to 4 steps (see Fig. \ref{fig:process-diagram}): (A) motion lab experiment, (B) filtering and preprocessing EMG data, (C) performing \ac{IK} and \ac{ID} to retrieve knee joint moments as training labels and data preparation before training, and (D) training and testing the network.

\begin{figure}[ht]
    \centering
    \includegraphics[width=0.9\textwidth]{img/ProcessDiagram}
    \caption{Flow diagram of the modelling procedure. The dashed box indicates procedure used only for training the \ac{NN}. (A) Subject walks in a motion lab collecting \ac{GRF} from force plates, marker trajectories, and muscle activity from \ac{sEMG} sensors; (B) The data is filtered and prepared for use in the models; (C) The \ac{GRF} and marker trajectories are used for \ac{IK} and \ac{ID} to retrieve knee joint moments; (D) During training the normalized \ac{EMG} is continuously fed through the \ac{NN}, the dashed arrows indicate feedback where the output is compared to the moments from (C) and the error used to update the \ac{NN}; (D) Once training is finished the \ac{EMG} can be fed through the \ac{NN} directly, producing knee joint moments to control motor torque in an exoskeleton}
    \label{fig:process-diagram}
\end{figure}

For the experiment (step A), a motion lab equipped with 10 Vicon Vantage V16 cameras and 3 AMTI OR6 \acp{FP} was used. 
The CGM2.3 markerset \textcite{Leboeuf2019} (see Fig. \ref{fig:cgm23-markerset}) was used without optional markers, except for RBAK which was included.
The aktos nano \ac{sEMG} sensors from myon were used for the \ac{EMG} data capture. 
The sensors were placed on (or near) 12 muscles (see table \ref{tab:muscle-names}) skin preparation and placement recommendations from the SENIAM group \cite{Hermens1999, Hermens2000}.

\begin{figure}
    \centering
    \includegraphics[width=0.9\textwidth]{img/CGM23_markerset2.png}
    \caption{CGM2.3 markerset}
    \label{fig:cgm23-markerset}
\end{figure}

All the data from the \ac{EMG} sensors (muscle activity), cameras (marker trajectories), and the \acp{FP} (\ac{GRF}) was collected and synchronized using the Vicon Nexus 2.8.2 system/software.
The marker trajectories were sampled at 100Hz using the cameras while the analog data (i.e. \ac{GRF} and \ac{EMG}) was sampled at 1000Hz.

In step B, before running \ac{IK} and \ac{ID}, the marker trajectories and the \ac{GRF} are filtered with a 4th order Butterworth lowpass filter with a cutoff frequency of $10Hz$. 
Same cutoff frequency is used to minimize an artifact introduced in the joint moment at heel strike, due to a spike in reaction force at that time \cite{Kristianslund2012}.
The \ac{EMG} was filtered with the "typical" steps as mentioned by \citeauthor{Clancy2016} \parencite[99]{Clancy2016}, but excluding the whitening step to keep the filtering real-time friendly.
Thus, the signal was noise and interference filtered using Butterworth IIR bandpass filter with cutoff frequencies $10-100Hz$. 
Then, for demodulation, smoothing, and relinearization, the RMS method was used where the smoothing was done with a moving average window of $20ms$. 
As \citeauthor{Clancy2016} point out, it is common to use a window of $100-250 ms$ but for the purpose of keeping the solution real-time friendly $20ms$ is chosen to not introduce too much delay.

The pyCGM2 package by \textcite{Leboeuf2019} was used to apply \ac{IK} and \ac{ID} on the motion and force data.
The \ac{ID} computations are based on anthropometric segment measurements according to \textcite{Dempster1955} and the iterative Newton-Euler equations as described by \textcite{Dumas2004}.

Introduce TensorFlow and the Keras API for the implementation of the \ac{NN}.
The training set for the \ac{NN} is normalized EMG data from regular walking.
The training labels are the corresponding joint torque values obtained from Vicon Nexus CGM2.

\section{Data collection and Inverse Dynamics}
Before running inverse dynamics on the model, the force plate and trajectory data is filtered. A 4th order Butterworth lowpass filter with a cutoff frequency of $10Hz$ is used here. Same cutoff frequency is used for both the force plate data and the trajectory data to minimize an artifact introduced in the joint moment at heel strike, due to a spike in reaction force at that time \cite{Kristianslund2012}.

\section{Surface-EMG filtering and feature extraction}
The filtering steps are \parencite[99]{Clancy2016}:
\begin{enumerate}
    \item Noise and interference filtering using Butterworth IIR bandpass filter with cutoff frequencies $10-100Hz$
    \item Temporal decorrelation (or whitening) to get rid of correlation from neighboring EMG channels
    \item Demodulation (this could be taking RMS?)
    \item Smoothing: moving average with a window of $20ms$, i.e. 20 samples. As \citeauthor{Clancy2016} points out, it is common to use a window of $100-250 ms$ but for the purpose of keeping the solution real-time friendly $20ms$ is chosen to not introduce too much delay.
    \item Relinearization
\end{enumerate}

\section{Training of the Neural Network}

\end{document}
