\section{Introduction}

Link things together with references. This is a reference to a section: \ref{sec:background}.

\subsection{Background}
\label{sec:background}
Present the background for the area. Give the context by explaining the parts that are needed to understand the degree project and thesis. (Still, keep in mind that this is an introductory part, which does not require too detailed description).

Use references\fancyfootnote{1}{You can also add footnotes if you want to clarify the content on the same page.}

Detailed description of the area should be moved to Chapter 2, where detailed information about background is given together with related work. 

This background presents background to writing a report in latex.

Look at sample table \ref{tab:sample-table-label} for a table sample.

\input{tables/sample-table}

Multiple images can be placed side by side like this:
\begin{figure}[ht]
	\centering 
	\subfloat[First subfigure]{\includegraphics[width=0.3\textwidth]{img/sample-image}
		\label{fig:sample-image:1}}
	\hfill
	\subfloat[Second subfigure]{\includegraphics[width=0.3\textwidth]{img/sample-image}
		\label{fig:sample-image:2}}
	\hfill
	\subfloat[Third subfigure]{\includegraphics[width=0.3\textwidth]{img/sample-image}
		\label{fig:sample-image:3}}
	\caption{\textit{General figure caption. The width might not add up to 1. Try make sure it adds up to 0.9 instead.}}
\end{figure}

\subsection{Problem}
Present the problems found in the area. Preferable use and end this section with a question as a problem statement.

Use references
Preferable, state the problem, to be solved, as a question. Do not use a question that can be answered with yes and/or no. 

\subsection{Purpose}
The purpose of the degree project/thesis is the purpose of the written material, i.e., the thesis. The thesis presents the work / discusses / illustrates and so on.

It is not “The project is about” even though this can be included in the purpose. If so, state the purpose of the project after purpose of the thesis).

\subsection{Goal}
The goal means the goal of the degree project. Present following: the goal(s), deliverables and results of the project. 

\subsubsection{Benefits, Ethics and Sustainability}
Describe who will benefit from the degree project, the ethical issues (what ethical problems can arise) and the sustainability aspects of the project.

Use references!

\subsection{Stakeholders}
Present the stakeholders for the degree project.

\subsection{Delimitations}
Explain the delimitations. These are all the things that could affect the study if they were examined and included in the degree project. 
Use references!

\subsection{Outline}
In text, describe what is presented in Chapters 2 and forward. Exclude the first chapter and references as well as appendix. 
